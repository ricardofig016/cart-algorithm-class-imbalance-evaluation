\documentclass{beamer}
%\usepackage[utf8]{inputenc}    
%\setbeamertemplate{navigation symbols}{}
\beamertemplatenavigationsymbolsempty{}
\setbeamertemplate{footline}[frame number]

\usetheme{Copenhagen}       
\title{Machine Learning I}
\author[Ricardo Figueiredo, Sérgio Cardoso]
{Ricardo Figueiredo up202105430 \and \\ Sérgio Cardoso up202107918}
\date{May 2025}

\begin{document}
\maketitle

\begin{frame}{Executive Summary}
    \begin{itemize}
        \item \textbf{Goals:} use a classification algorithm taught in classes 
        to obtain the accuracy of a set of benchmark datasets
        \item \textbf{Outline of the Approach:} modify an already 
        made implementation of the algorithm on existing open-source 
        code in order to make it robust to the data 
        characteristic in question.
        \item \textbf{Summary of Results:}
    \end{itemize}
\end{frame}

\begin{frame}{Selected Algorithm and Data Characteristic}
    We chose the CART algorithm due to its easy learning curve. 
    CART stands for Classification and Regression Tree which, in 
    itself, uses a Decision Tree to determine the accuracy of a dataset. 
    Decision Trees are among the most popular models in ML, as they are used 
    not only by CART, but also by other algorithms, such as ID3 (Iteractive 
    Dichotomiser 3) or C4.5, the latter being an extension of CART. \\ 
    \textbf{Gini index:} $Gini(D) = 1 - \sum_{i = 1}^{c} {p_{i}}^2$, where
    $p_{i}$ is the probability of class i.

\end{frame}

\begin{frame}{Selected Algorithm and Data Characteristic}
        Many classification algorithms are sensitive to certain data. For 
        instance, k-NN is sensitive to the presence of outliers, 
        SVM is sensitive to hyperparameter values. \\ 
        For this assignment, we were given the following three set of 
        benchmark datasets:
        \begin{enumerate}
            \item Noise or Outliers 
            \item Class Imbalance in Binary Classification 
            \item Multiclass Classification
        \end{enumerate}
        Given that we chose CART for this assignment and its 
        only
\end{frame}

\begin{frame}{Proposal}
\end{frame}

\begin{frame}{Emprical Study}
    \begin{itemize}
        \item \textbf{Experimental Setup:}
    \end{itemize}
\end{frame}

\begin{frame}{Emprical Study}
\end{frame}

\begin{frame}{Conclusions}
\end{frame}

\begin{frame}{References}
\end{frame}

\end{document}